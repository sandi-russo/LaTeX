\documentclass[aspectratio=169]{beamer}
\usepackage[utf8]{inputenc}
\usepackage[italian]{babel}
\usepackage{graphicx}
\usepackage{amsmath}
\usepackage{amssymb}
\usepackage{listings}
\usepackage{xcolor}
\usepackage{tikz}
\usepackage{fontawesome5}
\usepackage{colortbl}
\usepackage{comment}
\usetikzlibrary{arrows.meta,shadows,shapes.geometric,positioning}

\usetheme{metropolis}
\usecolortheme{default}

\definecolor{primaryDark}{RGB}{45,55,72}
\definecolor{accentBlue}{RGB}{66,153,225}
\definecolor{darkGray}{RGB}{74,85,104}
\definecolor{lightGray}{RGB}{247,250,252}
\definecolor{mediumGray}{RGB}{160,174,192}
\definecolor{successGreen}{RGB}{72,187,120}
\definecolor{warningRed}{RGB}{245,101,101}

\setbeamercolor{palette primary}{bg=primaryDark,fg=white}
\setbeamercolor{palette secondary}{bg=accentBlue,fg=white}
\setbeamercolor{palette tertiary}{bg=darkGray,fg=white}
\setbeamercolor{frametitle}{bg=primaryDark,fg=white}
\setbeamercolor{title}{bg=primaryDark,fg=white}
\setbeamercolor{block title}{bg=primaryDark,fg=white}
\setbeamercolor{block body}{bg=lightGray,fg=darkGray}
\setbeamercolor{alerted text}{fg=warningRed}
\setbeamercolor{example text}{fg=successGreen}

\setbeamertemplate{frame numbering}[fraction]
\setbeamertemplate{footline}{
  \leavevmode%
  \hbox{%
  \begin{beamercolorbox}[wd=.5\paperwidth,ht=2.5ex,dp=1ex,left,leftskip=1em]{author in head/foot}%
    \usebeamerfont{author in head/foot}\textbf{Sandi Russo} \hspace{0.3cm} | \hspace{0.3cm} \textit{UniME - Scienze Informatiche}
  \end{beamercolorbox}%
  \begin{beamercolorbox}[wd=.5\paperwidth,ht=2.5ex,dp=1ex,right,rightskip=1em]{date in head/foot}%
    \usebeamerfont{date in head/foot}
    \insertframenumber{} / \inserttotalframenumber
  \end{beamercolorbox}}%
  \vskip0pt%
}

\setbeamertemplate{navigation symbols}{}

\lstset{
    basicstyle=\ttfamily\footnotesize,
    keywordstyle=\color{accentBlue}\bfseries,
    commentstyle=\color{successGreen}\itshape,
    stringstyle=\color{warningRed},
    breaklines=true,
    backgroundcolor=\color{lightGray},
    frame=none,
    numbers=left,
    numberstyle=\tiny\color{mediumGray},
    xleftmargin=1.5em,
    framexleftmargin=1em,
    rulecolor=\color{primaryDark},
    showstringspaces=false,
    tabsize=2
}

\tikzset{
    modernbox/.style={
        rectangle, rounded corners=2pt, 
        draw=primaryDark, thick, 
        fill=lightGray,
        drop shadow={shadow xshift=1.5pt, shadow yshift=-1.5pt, opacity=0.25}
    },
    technode/.style={
        circle, draw=primaryDark, thick,
        minimum size=1.1cm,
        drop shadow={shadow xshift=1pt, shadow yshift=-1pt, opacity=0.3}
    },
    modernArrow/.style={
        ->, thick, >=stealth, 
        color=primaryDark
    }
}

\title[Graph ML con Oracle]{Knowledge Graph e Graph Machine Learning}
\subtitle{Analisi Comparativa e Implementazione con Oracle}
\author{Sandi Russo}
\institute{Università degli Studi di Messina\\Corso di Laurea in Scienze Informatiche}
\date{}

\begin{document}

{
\setbeamertemplate{footline}{}
\begin{frame}
    \vspace{1cm}
    \begin{center}
        {\Large\textbf{Knowledge Graph e Graph Machine Learning}}
        
        \vspace{0.3cm}
        
        {\large Analisi Comparativa e Implementazione con Oracle}
        
        \vspace{1cm}
        
        \begin{tikzpicture}
            \node[draw=primaryDark, thick, rounded corners=3pt, fill=lightGray!30, 
                  text width=9cm, align=center, inner sep=0.5cm] {
                \small
                \begin{tabular}{cl}
                    \faUser & \textbf{Sandi Russo} \\[0.15cm]
                    \faUniversity & Università degli Studi di Messina \\[0.15cm]
                    \faGraduationCap & Scienze Informatiche \\[0.15cm]
                    \faCalendar & Anno Accademico 2024/2025 \\[0.15cm]
                    \faChalkboardTeacher & Relatore: docente Antonio Celesti \\
                \end{tabular}
            };
        \end{tikzpicture}
    \end{center}
\end{frame}
}

\begin{frame}{Agenda}
    \tableofcontents
\end{frame}

\section{Introduzione al Machine Learning e ai Grafi}

\begin{frame}{Programmazione Tradizionale vs Machine Learning}
    \begin{columns}[T]
        \begin{column}{0.48\textwidth}
            \centering
            \textbf{\faCode \hspace{0.2cm} Programmazione Tradizionale}
            \vspace{0.2cm}
            
            \begin{tikzpicture}[node distance=1.4cm]
                \node (input) [modernbox, fill=accentBlue!20, minimum height=0.8cm] {\small \textbf{Dati}};
                \node (rules) [modernbox, fill=successGreen!20, below of=input, minimum height=0.8cm] {\small \textbf{Regole}};
                \node (output) [modernbox, fill=warningRed!20, below of=rules, minimum height=0.8cm] {\small \textbf{Output}};
                \draw[modernArrow] (input) -- (rules);
                \draw[modernArrow] (rules) -- (output);
            \end{tikzpicture}
            
            \vspace{0.2cm}
            \small
            \begin{itemize}
                \item[\faCogs] Logica esplicita
                \item[\faFileCode] Regole hard-coded
                \item[\faLock] Poco flessibile
            \end{itemize}
        \end{column}
        
        \begin{column}{0.48\textwidth}
            \centering
            \textbf{\faBrain \hspace{0.2cm} Machine Learning}
            \vspace{0.2cm}
            
            \begin{tikzpicture}[node distance=1.2cm]
                \node (input) [modernbox, fill=accentBlue!20, minimum height=0.8cm] {\small \textbf{Dati}};
                \node (labels) [modernbox, fill=warningRed!20, right of=input, xshift=0.5cm, minimum height=0.8cm] {\small \textbf{Labels}};
                \node (model) [modernbox, fill=successGreen!20, below of=input, xshift=0.75cm, yshift=-0.2cm, minimum height=0.8cm] {\small \textbf{Modello}};
                \draw[modernArrow] (input) -- (model);
                \draw[modernArrow] (labels) -- (model);
            \end{tikzpicture}
            
            \vspace{0.2cm}
            \small
            \begin{itemize}
                \item[\faRobot] Apprendimento automatico
                \item[\faChartLine] Pattern discovery
                \item[\faUnlock] Adattivo
            \end{itemize}
        \end{column}
    \end{columns}
\end{frame}

\begin{frame}{La Gerarchia dell'Intelligenza Artificiale}
    \centering
    \begin{tikzpicture}[scale=0.95]
        \draw[fill=accentBlue!10, thick, draw=accentBlue] (0,0) ellipse (4.2cm and 3.2cm);
        \node at (0,2.3) {\large \faLightbulb \hspace{0.2cm} \textbf{Intelligenza Artificiale}};
        
        \draw[fill=successGreen!10, thick, draw=successGreen] (0,-0.35) ellipse (3.3cm and 2.3cm);
        \node at (0,1.1) {\normalsize \faBrain \hspace{0.2cm} \textbf{Machine Learning}};
        
        \draw[fill=warningRed!10, thick, draw=warningRed] (0,-0.75) ellipse (2.4cm and 1.4cm);
        \node at (0,-0.2) {\small \faLayerGroup \hspace{0.2cm} \textbf{Deep Learning}};
        
        \draw[fill=primaryDark!20, thick, draw=primaryDark] (0,-1.4) ellipse (1.7cm and 0.75cm);
        \node at (0,-1.4) {\footnotesize \faProjectDiagram \hspace{0.2cm} \textbf{Graph ML}};
    \end{tikzpicture}
\end{frame}

\begin{frame}{Tipologie di Apprendimento}
    \centering
    \small
    \renewcommand{\arraystretch}{1.4}
    \begin{tabular}{|l|p{10cm}|}
        \hline
        \rowcolor{primaryDark!15}
        \textbf{Tipo} & \textbf{Descrizione} \\
        \hline
        \textbf{Supervised} & Dati + etichette $\rightarrow$ Es: riconoscimento immagini \\
        \hline
        \rowcolor{lightGray!50}
        \textbf{Unsupervised} & Solo dati, senza etichette $\rightarrow$ Es: clustering utenti \\
        \hline
        \textbf{Semi-supervised} & Mix di dati etichettati e non etichettati \\
        \hline
    \end{tabular}
\end{frame}


\section{Knowledge Graph e Graph Data Science}

\begin{frame}{Dal Grafo al Knowledge Graph}
    \begin{columns}[T]
        \begin{column}{0.48\textwidth}
            \textbf{\faCircle \hspace{0.2cm} Grafo Tradizionale}
            \small
            \begin{itemize}
                \item[\faCircle] Nodi omogenei
                \item[\faArrowRight] Archi omogenei
                \item[\faList] Struttura semplice
            \end{itemize}
        \end{column}
        
        \begin{column}{0.48\textwidth}
            \textbf{\faProjectDiagram \hspace{0.2cm} Knowledge Graph}
            \small
            \begin{itemize}
                \item[\faUsers] Nodi eterogenei
                \item[\faLink] Relazioni con proprietà
                \item[\faDatabase] Modellazione ricca
            \end{itemize}
        \end{column}
    \end{columns}
    
    \vspace{0.4cm}
    
    \begin{exampleblock}{Esempio Pratico}
        \centering
        \begin{tikzpicture}[
            node/.style={technode},
            rel/.style={modernArrow}
        ]
            \node[node, fill=accentBlue!30] (cane) at (0,0) {\faDog};
            \node[node, fill=successGreen!30] (uomo) at (3,1) {\faUser};
            \node[node, fill=warningRed!30] (gatti) at (3,-1) {\faCat};
            \node[node, fill=mediumGray!30] (cibo) at (6,0) {\faUtensils};
            
            \draw[rel] (cane) -- node[above, sloped, font=\tiny] {\textit{amico\_di}} (uomo);
            \draw[rel] (cane) -- node[below, sloped, font=\tiny] {\textit{odia}} (gatti);
            \draw[rel] (cane) -- node[above, sloped, font=\tiny] {\textit{mangia}} (cibo);
        \end{tikzpicture}
    \end{exampleblock}
\end{frame}

\begin{frame}{Graph Data Science: Capacità di Analisi}
    \begin{block}{\faProjectDiagram \hspace{0.2cm} Cosa Analizza la Graph Data Science}
        \small
        \begin{itemize}
            \item[\faChartLine] Evoluzione temporale di nodi e relazioni
            \item[\faStar] Identificazione dei nodi più influenti
            \item[\faNetworkWired] Interazioni e dinamiche tra gruppi
            \item[\faLightbulb] Pattern significativi emergenti
        \end{itemize}
    \end{block}
    
    \vspace{0.3cm}
    
    \begin{alertblock}{\faCheckCircle \hspace{0.2cm} Vantaggio Chiave}
        Il Graph Machine Learning \textbf{non è una black box}: la struttura dei grafi permette \textit{interpretabilità} dei risultati.
    \end{alertblock}
\end{frame}

\begin{frame}{Knowledge Graph Embedding}
    \textbf{Obiettivo}: Trasformare nodi e relazioni in vettori numerici
    
    $$\theta : V \rightarrow \mathbb{R}^d$$
    
    \vspace{0.15cm}
    
    \begin{columns}[T]
        \begin{column}{0.48\textwidth}
            \begin{block}{\faRandom \hspace{0.2cm} Random Walk}
                \small
                \begin{itemize}
                    \item Ispirato a Word2Vec
                    \item Percorsi casuali
                    \item Embedding \textit{generici}
                \end{itemize}
            \end{block}
        \end{column}
        
        \begin{column}{0.48\textwidth}
            \begin{block}{\faBrain \hspace{0.2cm} Neural Networks}
                \small
                \begin{itemize}
                    \item Comunicazione tra nodi
                    \item Scambio proprietà (hop)
                    \item Embedding \textit{task-specific}
                \end{itemize}
            \end{block}
        \end{column}
    \end{columns}
    
    \vspace{0.25cm}
    
    \begin{exampleblock}{Applicazioni}
        \centering
        \small
        \faCheck \hspace{0.15cm} Classificazione \hspace{0.4cm} 
        \faLink \hspace{0.15cm} Link Prediction \hspace{0.4cm}
        \faChartLine \hspace{0.15cm} Regressione
    \end{exampleblock}
\end{frame}
\section{Analisi Comparativa dei Database a Grafo}

\begin{frame}{Database a Grafo: Panoramica}
    \centering
    \scriptsize
    \renewcommand{\arraystretch}{1.2}
    \begin{tabular}{|l|c|c|c|c|}
        \hline
        \rowcolor{primaryDark!15}
        \textbf{Caratteristica} & \textbf{Neo4j} & \textbf{Oracle} & \textbf{ArangoDB} & \textbf{Neptune} \\
        \hline
        Versione gratuita & Community & Always Free & Community & Trial 30gg \\
        \hline
        \rowcolor{lightGray!50}
        GDS integrata & Sì (vasta) & Sì (PGX) & No & No \\
        \hline
        Linguaggi query & Cypher & PGQL/SQL & AQL & Gremlin \\
        \hline
        \rowcolor{lightGray!50}
        Multi-modello & No & No & Sì & No \\
        \hline
        ML Integration & Python & OML4Py & Python & SageMaker \\
        \hline
        \rowcolor{lightGray!50}
        Scalabilità & Alta & Altissima & Alta & Altissima \\
        \hline
        Curva & Media & Alta & Bassa & Alta \\
        \hline
    \end{tabular}
\end{frame}

\begin{frame}{Neo4j: Lo Standard di Riferimento}
    \begin{columns}[T]
        \begin{column}{0.48\textwidth}
            \begin{block}{\faThumbsUp \hspace{0.2cm} PRO}
                \small
                \begin{itemize}
                    \item[\textcolor{successGreen}{\faCheckCircle}] Community completa
                    \item[\textcolor{successGreen}{\faCheckCircle}] GDS vastissima
                    \item[\textcolor{successGreen}{\faCheckCircle}] Comunità enorme
                    \item[\textcolor{successGreen}{\faCheckCircle}] Docs eccellenti
                    \item[\textcolor{successGreen}{\faCheckCircle}] Cypher intuitivo
                \end{itemize}
            \end{block}
        \end{column}
        
        \begin{column}{0.48\textwidth}
            \begin{alertblock}{\faExclamationTriangle \hspace{0.2cm} CONTRO}
                \small
                \begin{itemize}
                    \item[\textcolor{warningRed}{\faTimesCircle}] Limitazioni free
                    \item[\textcolor{warningRed}{\faTimesCircle}] No multi-modello
                    \item[\textcolor{warningRed}{\faTimesCircle}] Costi Enterprise
                \end{itemize}
            \end{alertblock}
        \end{column}
    \end{columns}
    
    \vspace{0.4cm}
    
    \begin{exampleblock}{\faLightbulb \hspace{0.2cm} Caso d'uso ideale}
        \small
        Progetti con budget e supporto commerciale
    \end{exampleblock}
\end{frame}

\begin{frame}{ArangoDB: Flessibilità Multi-Modello}
    \begin{columns}[T]
        \begin{column}{0.48\textwidth}
            \begin{block}{\faThumbsUp \hspace{0.2cm} PRO}
                \small
                \begin{itemize}
                    \item[\textcolor{successGreen}{\faCheckCircle}] 100\% open-source
                    \item[\textcolor{successGreen}{\faCheckCircle}] Multi-modello
                    \item[\textcolor{successGreen}{\faCheckCircle}] AQL flessibile
                    \item[\textcolor{successGreen}{\faCheckCircle}] Leggero
                    \item[\textcolor{successGreen}{\faCheckCircle}] Python fluido
                \end{itemize}
            \end{block}
        \end{column}
        
        \begin{column}{0.48\textwidth}
            \begin{alertblock}{\faExclamationTriangle \hspace{0.2cm} CONTRO}
                \small
                \begin{itemize}
                    \item[\textcolor{warningRed}{\faTimesCircle}] No GDS integrata
                    \item[\textcolor{warningRed}{\faTimesCircle}] Comunità piccola
                    \item[\textcolor{warningRed}{\faTimesCircle}] Librerie esterne
                \end{itemize}
            \end{alertblock}
        \end{column}
    \end{columns}
    
    \vspace{0.4cm}
    
    \begin{exampleblock}{\faLightbulb \hspace{0.2cm} Caso d'uso ideale}
        \small
        Prototipazione rapida e flessibilità
    \end{exampleblock}
\end{frame}

\begin{frame}{Amazon Neptune: Cloud-Native}
    \begin{columns}[T]
        \begin{column}{0.48\textwidth}
            \begin{block}{\faThumbsUp \hspace{0.2cm} PRO}
                \small
                \begin{itemize}
                    \item[\textcolor{successGreen}{\faCheckCircle}] Scalabilità estrema
                    \item[\textcolor{successGreen}{\faCheckCircle}] Fully managed
                    \item[\textcolor{successGreen}{\faCheckCircle}] Gremlin + SPARQL
                    \item[\textcolor{successGreen}{\faCheckCircle}] AWS/SageMaker
                \end{itemize}
            \end{block}
        \end{column}
        
        \begin{column}{0.48\textwidth}
            \begin{alertblock}{\faExclamationTriangle \hspace{0.2cm} CONTRO}
                \small
                \begin{itemize}
                    \item[\textcolor{warningRed}{\faTimesCircle}] Trial 30 giorni
                    \item[\textcolor{warningRed}{\faTimesCircle}] Costi elevati
                    \item[\textcolor{warningRed}{\faTimesCircle}] No GDS integrata
                    \item[\textcolor{warningRed}{\faTimesCircle}] Lock-in AWS
                \end{itemize}
            \end{alertblock}
        \end{column}
    \end{columns}
    
    \vspace{0.4cm}
    
    \begin{alertblock}{\faBan \hspace{0.2cm} Limitazione critica}
        \small
        \textbf{Inadatto per progetti universitari} (pay-per-use)
    \end{alertblock}
\end{frame}

\section{Oracle Graph: La Scelta Motivata}

\begin{frame}{Oracle Graph Server and Client: Panoramica}
    \begin{columns}[T]
        \begin{column}{0.48\textwidth}
            \begin{block}{\faThumbsUp \hspace{0.2cm} PRO}
                \small
                \begin{itemize}
                    \item[\textcolor{successGreen}{\faCheckCircle}] Always Free Tier
                    \item[\textcolor{successGreen}{\faCheckCircle}] PGX potentissimo
                    \item[\textcolor{successGreen}{\faCheckCircle}] 60+ algoritmi
                    \item[\textcolor{successGreen}{\faCheckCircle}] PGQL SQL-like
                    \item[\textcolor{successGreen}{\faCheckCircle}] OML4Py nativo
                \end{itemize}
            \end{block}
        \end{column}
        
        \begin{column}{0.48\textwidth}
            \begin{alertblock}{\faExclamationTriangle \hspace{0.2cm} CONTRO}
                \small
                \begin{itemize}
                    \item[\textcolor{warningRed}{\faTimesCircle}] Più risorse
                    \item[\textcolor{warningRed}{\faTimesCircle}] Curva ripida
                    \item[\textcolor{warningRed}{\faTimesCircle}] Meno agile
                    \item[\textcolor{warningRed}{\faTimesCircle}] Setup complesso
                \end{itemize}
            \end{alertblock}
        \end{column}
    \end{columns}
    
    \vspace{0.3cm}
    
    \begin{block}{\faCogs \hspace{0.2cm} Componenti Chiave}
        \centering
        \small
        \faDatabase \hspace{0.1cm} Oracle DB \hspace{0.5cm}
        \faServer \hspace{0.1cm} PGX Server \hspace{0.5cm}
        \faCode \hspace{0.1cm} PGQL \hspace{0.5cm}
        \faPython \hspace{0.1cm} OML4Py
    \end{block}
\end{frame}

\begin{frame}{Architettura Oracle Graph}
    \centering
    \begin{tikzpicture}[
        box/.style={modernbox, minimum width=2.8cm, minimum height=1cm, align=center, font=\small},
        arrow/.style={modernArrow}
    ]
        \node[box, fill=accentBlue!20] (db) at (0,0) {\faDatabase \\ \textbf{Oracle DB}};
        
        \node[box, fill=successGreen!20] (pgx) at (0,1.8) {\faServer \\ \textbf{PGX Server}};
        
        \node[box, fill=warningRed!20] (pgql) at (-3,3.6) {\faCode \\ \textbf{PGQL}};
        \node[box, fill=mediumGray!20] (oml4py) at (0,3.6) {\faPython \\ \textbf{OML4Py}};
        \node[box, fill=primaryDark!20] (studio) at (3,3.6) {\faChartBar \\ \textbf{Studio}};
        
        \draw[arrow] (db) -- (pgx);
        \draw[arrow] (pgx) -- (pgql);
        \draw[arrow] (pgx) -- (oml4py);
        \draw[arrow] (pgx) -- (studio);
    \end{tikzpicture}
\end{frame}

\begin{frame}{Perché Oracle per questo Progetto?}
    \small
    \begin{enumerate}
        \item[\faDollarSign] \textbf{Costo Zero per Ricerca/Didattica}
        \begin{itemize}
            \item Always Free Tier su Oracle Cloud
            \item Download gratuito per uso accademico
        \end{itemize}
        
        \item[\faRocket] \textbf{Potenza Analitica Superiore}
        \begin{itemize}
            \item PGX ottimizzato per grafi massivi
            \item Performance comparabili a Neo4j Enterprise
        \end{itemize}
        
        \item[\faPython] \textbf{Integrazione ML Nativa}
        \begin{itemize}
            \item OML4Py: ML direttamente su grafi
            \item PyPGX: embedding e GNN out-of-the-box
        \end{itemize}
        
        \item[\faGraduationCap] \textbf{Competenze Trasferibili}
        \begin{itemize}
            \item PGQL simile a SQL
            \item Esperienza enterprise-ready
        \end{itemize}
    \end{enumerate}
\end{frame}

\begin{frame}{Oracle vs Neo4j: Confronto Tecnico}
    \centering
    \scriptsize
    \renewcommand{\arraystretch}{1.15}
    \begin{tabular}{|p{2.8cm}|p{4.2cm}|p{4.2cm}|}
        \hline
        \rowcolor{primaryDark!15}
        \textbf{Aspetto} & \textbf{Neo4j Community} & \textbf{Oracle (Free)} \\
        \hline
        Costo & Gratis (limitato) & Gratis (completo) \\
        \hline
        \rowcolor{lightGray!50}
        Algoritmi GDS & 65+ (50 Enterprise) & 60+ (tutti) \\
        \hline
        Query & Cypher & PGQL (SQL-like) \\
        \hline
        \rowcolor{lightGray!50}
        ML & Python (esterno) & OML4Py (nativo) \\
        \hline
        Performance & Grafi medi & Grafi massivi \\
        \hline
        \rowcolor{lightGray!50}
        Deploy & Docker/Cloud & Docker/OCI \\
        \hline
        Docs & Vastissima & Ottima \\
        \hline
        \rowcolor{lightGray!50}
        Curva & Media & Alta \\
        \hline
        \rowcolor{successGreen!15}
        \textbf{Vantaggio} & Ecosistema & \textbf{Potenza} \\
        \hline
    \end{tabular}
\end{frame}

\begin{frame}{Oracle vs ArangoDB: Confronto Tecnico}
    \centering
    \scriptsize
    \renewcommand{\arraystretch}{1.15}
    \begin{tabular}{|p{2.8cm}|p{4.2cm}|p{4.2cm}|}
        \hline
        \rowcolor{primaryDark!15}
        \textbf{Aspetto} & \textbf{ArangoDB} & \textbf{Oracle} \\
        \hline
        Multi-Modello & Sì (docs+grafi) & No (solo grafi) \\
        \hline
        \rowcolor{lightGray!50}
        GDS & No & Sì (PGX) \\
        \hline
        Peso & \textasciitilde500MB & \textasciitilde2GB \\
        \hline
        \rowcolor{lightGray!50}
        Query & AQL & PGQL \\
        \hline
        Algoritmi & 0 (Python) & 60+ (nativi) \\
        \hline
        \rowcolor{lightGray!50}
        Embedding & Esterni & PyPGX \\
        \hline
        Setup & Rapido & Complesso \\
        \hline
        \rowcolor{lightGray!50}
        Community & Crescente & Enterprise \\
        \hline
        \rowcolor{successGreen!15}
        \textbf{Vantaggio} & Agilità & \textbf{Analisi} \\
        \hline
    \end{tabular}
\end{frame}

\section{Implementazione e Setup}

\begin{frame}[fragile]{Setup Oracle: Docker Quick Start}
    \small
    \textbf{\faDocker \hspace{0.2cm} Passo 1}: Pull immagine
    \begin{lstlisting}[language=bash, basicstyle=\ttfamily\tiny]
docker pull container-registry.oracle.com/database/graph-quickstart:latest
\end{lstlisting}
    
    \vspace{0.2cm}
    
    \textbf{\faPlay \hspace{0.2cm} Passo 2}: Avvio container
    \begin{lstlisting}[language=bash, basicstyle=\ttfamily\tiny]
docker run -d --name oracle-graph-ml \
  -p 1521:1521 -p 7007:7007 \
  -e ORACLE_PWD=Password234 \
  container-registry.oracle.com/database/graph-quickstart:latest
\end{lstlisting}
    
    \vspace{0.2cm}
    
    \begin{alertblock}{\faNetworkWired \hspace{0.2cm} Porte Esposte}
        \small
        \textbf{1521}: Oracle Database \hspace{0.5cm}
        \textbf{7007}: Graph Studio
    \end{alertblock}
\end{frame}

\begin{frame}[fragile]{Connessione al Database}
    \small
    \textbf{\faKey \hspace{0.2cm} Credenziali}:
    
    \begin{lstlisting}[language=bash, basicstyle=\ttfamily\tiny]
# Graph Studio (https://localhost:7007/ui)
Username: GRAPHUSER
Password: Password234

# SQL Developer
Username: SYSTEM
Password: Password234
Hostname: localhost
Port: 1521
Service Name: FREEPDB1
\end{lstlisting}
    
    \vspace{0.2cm}
    
    \textbf{\faPython \hspace{0.2cm} Librerie Python}:
    \begin{lstlisting}[language=bash, basicstyle=\ttfamily\tiny]
pip install oracledb oml4py pypgx
\end{lstlisting}
\end{frame}

\begin{comment}
\begin{frame}[fragile]{Esempio: Creazione Property Graph}
    \begin{lstlisting}[language=SQL, basicstyle=\ttfamily\tiny]
CREATE PROPERTY GRAPH social_network
  VERTEX TABLES (
    users KEY (user_id)
      PROPERTIES (user_id, name, age)
  )
  EDGE TABLES (
    friendships KEY (from_user, to_user)
      SOURCE KEY (from_user) REFERENCES users
      DESTINATION KEY (to_user) REFERENCES users
      PROPERTIES (since_date, strength)
  );
\end{lstlisting}
\end{frame}

\begin{frame}[fragile]{Esempio: Query PGQL}
    \begin{lstlisting}[language=SQL, basicstyle=\ttfamily\tiny]
-- Trova gli amici degli amici (2-hop)
SELECT u1.name AS person, u3.name AS friend_of_friend
FROM MATCH (u1)-[:friendships]->(u2)-[:friendships]->(u3)
WHERE u1.user_id = 123 AND u3.user_id <> u1.user_id;

-- Calcola PageRank
SELECT user_id, name, 
       GRAPH_ALGORITHM.pagerank(social_network) AS rank
FROM users ORDER BY rank DESC LIMIT 10;
\end{lstlisting}
\end{frame}

\begin{frame}[fragile]{Esempio: Python + OML4Py}
    \begin{lstlisting}[language=Python, basicstyle=\ttfamily\tiny]
import oracledb
from oml import sync
from pypgx import get_instance

conn = oracledb.connect(user="GRAPHUSER", password="Password234",
                        dsn="localhost:1521/FREEPDB1")

sync(conn, table="USERS", schema="GRAPHUSER")

session = get_instance().create_session("my_session")
graph = session.read_graph_by_name("social_network")

model = graph.create_deepwalk_model(
    walks_per_node=10, walk_length=5, embedding_dim=128
)
embeddings = model.get_node_embeddings()
\end{lstlisting}
\end{frame}
\end{comment}

\section{Conclusioni e Sviluppi Futuri}

\begin{frame}{Sintesi della Scelta}
    \begin{block}{\faCheckCircle \hspace{0.2cm} Perché Oracle Graph?}
        \vspace{0.2cm}
        \small
        \begin{columns}[T]
            \begin{column}{0.48\textwidth}
                \begin{itemize}
                    \item[\hspace{0.1em}\faDollarSign\hspace{0.1em}] \textbf{Gratuità}: Always Free Tier
                    \item[\hspace{0.1em}\faRocket\hspace{0.1em}] \textbf{Potenza}: PGX supera Neo4j Community
                \end{itemize}
            \end{column}
            \begin{column}{0.48\textwidth}
                \begin{itemize}
                    \item[\hspace{0.1em}\faPuzzlePiece\hspace{0.1em}] \textbf{Completezza}: ML nativo
                    \item[\hspace{0.1em}\faGraduationCap\hspace{0.1em}] \textbf{Formazione}: Esperienza enterprise
                \end{itemize}
            \end{column}
        \end{columns}
    \end{block}
    
    \vspace{0.3cm}
    
    \begin{alertblock}{\faBalanceScale \hspace{0.2cm} Trade-off Accettato}
        \vspace{0.2cm}
        \small
        Complessità iniziale compensata da:
        \begin{columns}[T]
            \begin{column}{0.48\textwidth}
                \begin{itemize}
                    \item[\hspace{0.1em}\faChartLine\hspace{0.1em}] Capacità analitiche superiori
                    \item[\hspace{0.1em}\faDollarSign\hspace{0.1em}] Nessun costo (vs Neptune)
                \end{itemize}
            \end{column}
            \begin{column}{0.48\textwidth}
                \begin{itemize}
                    \item[\hspace{0.1em}\faCogs\hspace{0.1em}] GDS integrata (vs ArangoDB)
                    \item[\hspace{0.1em}\faUnlock\hspace{0.1em}] Funzionalità complete (vs Neo4j)
                \end{itemize}
            \end{column}
        \end{columns}
    \end{alertblock}
\end{frame}

\begin{frame}{Limitazioni e Considerazioni}
    \small
    \textbf{\faExclamationTriangle \hspace{0.2cm} Punti di attenzione}:
    \begin{itemize}
        \item[\faMemory] \textbf{Risorse}: 4GB RAM vs 2GB
        \item[\faCogs] \textbf{Setup}: DB + PGX separati
        \item[\faBook] \textbf{Docs}: Meno beginner-friendly
        \item[\faUsers] \textbf{Community}: Più enterprise
    \end{itemize}
    
    \vspace{0.3cm}
    
    \textbf{\faLightbulb \hspace{0.2cm} Alternative}:
    \begin{itemize}
        \item[\faCircle] \textbf{Neo4j}: Ecosistema maturo
        \item[\faCircle] \textbf{ArangoDB}: Setup minimale
        \item[\faCircle] \textbf{Neptune}: Ambiente AWS
    \end{itemize}
\end{frame}

\begin{frame}{Sviluppi Futuri}
    \small
    \textbf{\faRoad \hspace{0.2cm} Estensioni possibili}:
    \begin{enumerate}
        \item[\faChartBar] \textbf{Benchmark}
        \begin{itemize}
            \item Performance su dataset reali
            \item Analisi tempi/memoria
        \end{itemize}
        
        \item[\faBrain] \textbf{ML Avanzato}
        \begin{itemize}
            \item GNN con PyPGX
            \item Link prediction complessa
        \end{itemize}
        
        \item[\faGlobe] \textbf{Web Semantico}
        \begin{itemize}
            \item Import RDF/OWL
            \item Reasoning ontologie
        \end{itemize}
        
        \item[\faCloud] \textbf{Produzione}
        \begin{itemize}
            \item Deploy su OCI
            \item API REST
        \end{itemize}
    \end{enumerate}
\end{frame}

\begin{frame}{Risorse e Riferimenti}
    \footnotesize
    \setlength{\itemsep}{2pt}
    \setlength{\parskip}{0pt}
    
    \textbf{\faBook \hspace{0.2cm} Oracle}:
    \begin{itemize}
        \setlength{\itemsep}{1pt}
        \item Property Graph: \texttt{docs.oracle.com/.../property-graph/}
        \item OML4Py: \texttt{docs.oracle.com/.../oml4py/}
        \item Graph Studio: \texttt{docs.oracle.com/.../graph-studio...}
    \end{itemize}
    
    \textbf{\faDatabase \hspace{0.2cm} Alternative}:
    \begin{itemize}
        \setlength{\itemsep}{1pt}
        \item Neo4j: \texttt{graphacademy.neo4j.com}
        \item ArangoDB: \texttt{arangodb.com/docs/}
        \item Neptune: \texttt{aws.amazon.com/neptune/}
    \end{itemize}
    
    \textbf{\faPython \hspace{0.2cm} ML}:
    \begin{itemize}
        \setlength{\itemsep}{1pt}
        \item PyTorch Geometric: \texttt{pytorch-geometric.readthedocs.io}
        \item DGL: \texttt{dgl.ai}
    \end{itemize}
\end{frame}

{
\setbeamertemplate{footline}{}
\begin{frame}[standout]
    \Huge Grazie per l'attenzione!
\end{frame}
}

\end{document}