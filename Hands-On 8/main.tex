\documentclass[a4paper, 12pt]{scrartcl}
\usepackage[utf8]{inputenc} % supporto caratteri UTF-8
\usepackage[T1]{fontenc} % migliore gestione dei font
\usepackage{newtxtext, newtxmath} % Font Times New Roman-like
\usepackage[main=italian, english]{babel} % imposta l'italiano come lingua principale
\usepackage{graphicx} % per l'inserimento di immagini
\usepackage{hyperref} % per link nell'indice
\usepackage{listings} % per codice sorgente
\usepackage{xcolor} % per colorare il codice
\usepackage{amsmath} % per formule matematiche complesse
\usepackage{tikz} % per i disegni
\usetikzlibrary{positioning} % per una migliore gestione delle posizioni degli elementi
\usetikzlibrary{decorations.pathreplacing} % per le graffe
\usetikzlibrary{shapes.geometric, arrows}
\usepackage{caption}
\usepackage{subcaption}

% formattazione codice sorgente
\lstset{
    language=C,
    basicstyle=\ttfamily\footnotesize,
    keywordstyle=\color{blue},
    commentstyle=\color{green!50!black},
    stringstyle=\color{red},
    numbers=left,
    numberstyle=\tiny,
    stepnumber=1,
    breaklines=true,
    frame=single,
    captionpos=b
}

% informazioni sulla relazione
\title{Hands-On 8}
\subtitle{MULTITHREAD}
\author{Sandi Russo \\ Corso di Laurea: Scienze Informatiche} % Nome e CdL
\date{\today}

\begin{document}

\maketitle % Genera la copertina con titolo, sottotitolo e autore

\newpage
\tableofcontents % Inserisce l'indice automatico
\newpage

\section{Introduzione}
In questo documento andremo ad analizzare il funzionamento e l'implementazione dell'algoritmo di calcolo dei numeri primi utilizzando il multithreading in C. Il multithreading consente di suddividere il lavoro tra più thread, migliorando l'efficienza e le prestazioni del programma.
\\
Un thread, o \textit{filo} di esecuzione, rappresenta il più piccolo insieme di istruzioni che possono essere gestite in modo indipendente da un \textit{scheduler} di un sistema operativo. Ogni programma ha almeno un thread, il \textit{main thread}, che viene creato automaticamente all'avvio del programma. Il multithreading è una tecnica che permette di eseguire più thread contemporaneamente, sfruttando al meglio le capacità di elaborazione parallela delle moderne CPU multi-core.
\\
Lavorare con i thread in C richiede l'uso della libreria \texttt{pthread}, che fornisce le funzioni necessarie per creare, gestire e sincronizzare i thread. La libreria \texttt{pthread} è conforme allo standard POSIX (Portable Operating System Interface) e permette di scrivere programmi portabili su diverse piattaforme Unix-like.

\section{Definizione del Problema}
Il problema che andremo a risolvere riguarda il calcolo dei numeri primi fino a un milione utilizzando quattro thread. L'obiettivo è sfruttare il multithreading per accelerare il processo di verifica dei numeri primi, distribuendo il carico di lavoro su più core della CPU.

\section{Metodologia}
In questa sezione, descriviamo la metodologia utilizzata per implementare e testare il codice di calcolo dei numeri primi utilizzando il multithreading. Il codice è stato sviluppato per dividere il range dei numeri da verificare tra i thread disponibili.

\subsection{Descrizione del Codice}
Il codice è suddiviso in diverse parti principali:
\begin{enumerate}
    \item \textbf{Definizione delle Costanti e delle Funzioni}: Definiamo i valori costanti e implementiamo la funzione per verificare se un numero è primo.
    \item \textbf{Funzione di Controllo dei Numeri Primi}: Implementiamo la funzione che ogni thread eseguirà per controllare i numeri primi nel proprio range.
    \item \textbf{Creazione e Gestione dei Thread}: Creiamo i thread, assegniamo a ciascuno un range di numeri da controllare e attendiamo la loro terminazione.
\end{enumerate}

\subsection{Test}
Per verificare il corretto funzionamento del programma, abbiamo eseguito il codice su una macchina con supporto per il multithreading e confrontato i risultati con quelli ottenuti da un programma sequenziale.

\section{Presentazione dei Risultati}
I risultati ottenuti dai test del codice multithreading sono stati soddisfacenti. Di seguito sono riportati alcuni esempi di numeri primi trovati:

\begin{verbatim}
Il numero 2 è primo
Il numero 3 è primo
Il numero 5 è primo
... (output abbreviato)
Il numero 999983 è primo
\end{verbatim}

Il programma ha dimostrato di essere in grado di trovare correttamente i numeri primi fino a un milione, distribuendo il carico di lavoro tra i thread. Inoltre, il tempo di esecuzione è stato notevolmente ridotto grazie all'uso del multithreading.

\subsection{Descrizione Dettagliata del Codice}
Qui descriviamo il funzionamento del codice passo dopo passo, utilizzando il pacchetto \texttt{listings} per includere i frammenti di codice.

\begin{lstlisting}[caption={Inclusione delle librerie necessarie}]
#include <stdio.h>
#include <pthread.h>
#include <stdbool.h>
\end{lstlisting}
Le librerie incluse forniscono le funzioni necessarie per la gestione dei thread, la manipolazione dei dati e altre funzionalità di sistema.

\begin{lstlisting}[caption={Definizione di costanti per il range di numeri e il numero di thread}]
#define MAX_NUMBER 1000000
#define MAX_THREAD 4
\end{lstlisting}
Definiamo il limite massimo del range di numeri da verificare e il numero di thread da utilizzare.

\begin{lstlisting}[caption={Funzione per verificare se un numero è primo}]
bool is_prime(int n) {
    if (n <= 1) return false;
    if (n == 2) return true;
    if (n % 2 == 0) return false;
    for (int i = 3; i * i <= n; i += 2) {
        if (n % i == 0) return false;}
    return true;}
\end{lstlisting}
Implementiamo la funzione \texttt{is\_prime()} che verifica se un numero è primo. La funzione utilizza un algoritmo efficiente per ridurre il numero di divisioni necessarie.

\begin{lstlisting}[caption={Funzione eseguita dai thread per controllare i numeri primi}]
void* check_primes(void* arg) {
    int start = *(int*)arg;
    int end = start + (MAX_NUMBER / MAX_THREAD);
    
    for (int i = start; i < end; i++) {
        if (is_prime(i)) {
            printf("Il numero %d e' primo\n", i);
        }
    }

    return NULL;
}
\end{lstlisting}
La funzione \texttt{check\_primes()} viene eseguita da ciascun thread. Ogni thread verifica i numeri primi nel proprio range e stampa i numeri primi trovati.

\begin{lstlisting}[caption={Funzione principale per la creazione e gestione dei thread}]
int main() {
    pthread_t threads[MAX_THREAD];
    int range[MAX_THREAD];

    for (int i = 0; i < MAX_THREAD; i++) {
        range[i] = i * (MAX_NUMBER / MAX_THREAD);
        
        // Creazione del thread
        int result = pthread_create(&threads[i], NULL, check_primes, &range[i]);
        
        if(result != 0) {
            perror("Errore nella creazione del thread");
            return 1;
        }
    }
    
    for (int i = 0; i < MAX_THREAD; i++) {
        pthread_join(threads[i], NULL);
    }
    
    printf("Tutti i thread sono terminati.\n");
    return 0;
}
\end{lstlisting}
Nella funzione \texttt{main()} creiamo i thread, assegnando a ciascuno un range di numeri da controllare. Utilizziamo \texttt{pthread\_create()} per creare i thread e \texttt{pthread\_join()} per attendere la loro terminazione. Alla fine del programma, stampiamo un messaggio indicando che tutti i thread sono terminati.

\section{Conclusione}
In questa relazione abbiamo descritto l'implementazione di un programma in C per il calcolo dei numeri primi utilizzando il multithreading. Abbiamo illustrato i vantaggi dell'uso del multithreading in termini di efficienza e prestazioni, suddividendo il carico di lavoro tra i thread. La metodologia adottata per lo sviluppo e il test del codice ha garantito il corretto funzionamento del programma, che è risultato in grado di trovare numeri primi fino a un milione in modo affidabile e veloce. Infine, i risultati dei test hanno confermato l'efficacia del programma nel contesto del calcolo parallelo.
\end{document}