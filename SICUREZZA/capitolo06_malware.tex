\chapter{Malware}

\section{Introduzione}

Il malware è un programma inserito in un sistema, di solito in modo nascosto, con l'intento di compromettere la confidenzialità, l'integrità o la disponibilità dei dati della vittima.

\subsection{Classificazione dei Malware}

Il malware può essere classificato in due grandi categorie: in base ai meccanismi di propagazione e alle azioni eseguite (o payload) una volta raggiunto il bersaglio.

\subsection{Propagazione dei Malware}

Il malware può propagarsi aggiungendo a programmi presenti nel sistema, porzioni di codice malevolo. La prima categoria di propagazione del malware riguarda frammenti di software parassita che si attaccano a contenuti eseguibili esistenti. Il frammento può essere codice macchina che infetta un'applicazione, un programma di utilità o persino il codice utilizzato per l'avvio del sistema. Più recentemente, il frammento è stato rappresentato da codice di scripting, utilizzato in file come documenti Word, fogli di calcolo Excel o documenti PDF.

\section{Tipi di Malware}

Il malware comprende una varietà di software dannosi progettati per compromettere la sicurezza, la privacy o l'integrità dei sistemi informatici. Di seguito sono presentati i principali tipi di malware con una descrizione dettagliata delle loro caratteristiche e modalità di azione.

\subsection{Advanced Persistent Threat (APT)}

Un APT è una minaccia avanzata e persistente che utilizza diverse forme di malware per attaccare un obiettivo specifico in modo continuo e discreto per un lungo periodo. Gli attacchi APT sono spesso sofisticati, orchestrati da gruppi di hacker organizzati, e mirano a ottenere l'accesso prolungato a una rete per rubare dati sensibili o sabotare le operazioni del bersaglio.

\subsection{Adware}

L'adware è un tipo di software che visualizza o scarica pubblicità indesiderata, spesso sotto forma di pop-up o reindirizzamenti a siti web commerciali. Sebbene alcune forme di adware possano essere relativamente innocue, altre possono compromettere l'esperienza dell'utente, raccogliere informazioni sulla navigazione e risultare difficili da rimuovere.

\subsection{Exploit Kit}

Un exploit kit è un insieme di strumenti che automatizza la creazione e la distribuzione di nuovo malware sfruttando vulnerabilità software note. Questi kit offrono meccanismi di propagazione sofisticati e moduli payload che anche utenti inesperti possono combinare e distribuire. Un esempio di exploit kit è il famoso Zeus, noto per il furto di credenziali bancarie. Altri kit, come Blackhole, Sakura e Phoenix, sono stati ampiamente utilizzati per generare attacchi complessi.

\subsection{Backdoor}

Le backdoor sono meccanismi che consentono di aggirare i normali controlli di sicurezza, garantendo un accesso non autorizzato a un sistema compromesso o a funzionalità di un programma. Le backdoor possono essere installate da attaccanti per ottenere un punto di accesso permanente, permettendo loro di eludere le protezioni di sicurezza.

\subsection{Exploits}

Gli exploit sono pezzi di codice progettati per sfruttare specifiche vulnerabilità nei software. Gli attacchi che utilizzano exploit possono danneggiare i sistemi, eseguire codice non autorizzato o accedere a dati sensibili.

\subsection{Flooders (DoS Client)}

Questi strumenti vengono utilizzati per generare un grande volume di traffico o dati al fine di inondare e sovraccaricare un sistema informatico o una rete. Il risultato è un attacco denial-of-service (DoS), che rende inaccessibili le risorse del sistema.

\subsection{Keyloggers}

I keylogger monitorano e registrano ogni tasto premuto su un sistema compromesso, fornendo agli attaccanti dati sensibili come password, numeri di carta di credito e altre informazioni riservate.

\subsection{Logic Bomb}

Le logic bomb sono malware programmati per rimanere inattivi fino al verificarsi di una specifica condizione predefinita, come una data o un evento. Una volta attivate, possono eseguire azioni dannose, come cancellare dati o disattivare sistemi.

\subsection{Macro Virus}

I macro virus utilizzano codice o script incorporati in documenti, che si attivano quando il documento viene aperto o modificato. Sono particolarmente diffusi in documenti creati con software come Word o Excel.

\subsection{Mobile Code}

Il mobile code consiste in script o software che possono essere trasferiti ed eseguiti su diverse piattaforme senza modifiche. Esempi comuni includono JavaScript e ActiveX, spesso usati nelle pagine web.

\subsection{Rootkit}

I rootkit sono insiemi di strumenti progettati per ottenere e mantenere il controllo root (amministratore) su un sistema senza essere rilevati. Possono nascondere la presenza di altri malware e modificare il funzionamento del sistema.

\subsection{Spyware}

Lo spyware raccoglie informazioni dal computer dell'utente senza il suo consenso e le trasmette a una terza parte. Questo software può monitorare la digitazione, registrare dati sullo schermo o analizzare i file del sistema.

\subsection{Trojan}

I trojan appaiono come programmi utili ma nascondono una funzione dannosa. Eludono i meccanismi di sicurezza sfruttando le autorizzazioni legittime del sistema e possono portare all'installazione di altri malware.

\subsection{Virus}

Un virus è un tipo di malware che, una volta eseguito, tenta di replicarsi inserendosi in altri file o programmi eseguibili. Quando questi file vengono avviati, il virus si attiva e può causare danni o ulteriori infezioni.

\subsection{Worm}

Un worm è un programma che si replica autonomamente e si diffonde ad altri host in rete sfruttando vulnerabilità software. I worm possono propagarsi rapidamente, causando disagi su vasta scala.

\subsection{Zombie, Bot}

Un bot o zombie è un programma che, una volta installato su un sistema compromesso, può essere utilizzato per lanciare attacchi contro altri sistemi, spesso coordinato da un server di comando e controllo in una botnet.

\section{Virus}

\subsection{Introduzione}

Un virus informatico è un software che infetta altri programmi, modificandoli. La modifica include l'iniezione di codice che consente al virus di replicarsi e infettare altri contenuti.

Come i virus biologici, un virus informatico contiene il ``codice istruttivo'' per replicarsi. Ogni volta che il computer infetto entra in contatto con un codice non infetto, il virus si trasferisce in questo nuovo luogo, diffondendosi da computer a computer. La diffusione può avvenire tramite supporti fisici come dischi o chiavette USB, oppure tramite reti, dove documenti, applicazioni e servizi di sistema condivisi facilitano la propagazione.

\subsection{Struttura Generale}

Un virus è costituito da tre parti:

\begin{description}
    \item[Meccanismo di infezione] Il mezzo con cui un virus si propaga, permettendogli di replicarsi. Il meccanismo viene anche chiamato vettore di infezione.
    \item[Trigger] L'evento che determina quando il payload viene attivato o consegnato, talvolta noto come bomba logica.
    \item[Payload] Ciò che il virus fa, oltre a diffondersi. Il payload può implicare un danno oppure un'attività benigna ma percettibile.
\end{description}

Una volta avviato, il virus cerca di infettare altri file eseguibili non infetti. Se vengono soddisfatte determinate condizioni, il virus esegue il suo payload e poi trasferisce il controllo al programma originale.

\subsection{Compressione del File Infetto}

Un virus può anche comprimere il file eseguibile infetto per evitare che la sua dimensione più grande rispetto al file originale non infetto possa farne scoprire la modifica. Questo processo di compressione e decompressione consente al virus di rimanere nascosto durante l'esecuzione del programma.

\subsection{Fasi di un Virus}

Un virus tipicamente attraversa le seguenti quattro fasi:

\begin{enumerate}
    \item \textbf{Fase dormiente:} il virus è inattivo. Al termine di questa fase il virus verrà attivato da qualche evento. Non tutti i virus prevedono questa fase.
    \item \textbf{Fase di propagazione:} il virus inserisce una copia di se stesso in altri programmi o in determinate aree di sistema del disco. La copia potrebbe non essere identica alla versione che si sta propagando in quanto i virus spesso mutano per sfuggire al rilevamento. Ogni programma infetto conterrà a questo punto un clone del virus, che a sua volta entrerà in una fase di propagazione.
    \item \textbf{Fase di attivazione:} il virus viene attivato per svolgere la funzione cui è destinato. Come per la fase dormiente, la fase di attivazione può essere scatenata da numerosi eventi di sistema, compreso il conteggio del numero di volte che questa copia del virus ha creato copie di se stessa.
    \item \textbf{Fase di esecuzione:} viene espletata la funzione. La funzione può essere innocua, come un messaggio sullo schermo, o dannosa, come la distruzione di programmi e di file di dati.
\end{enumerate}

\subsection{Classificazione dei Virus}

Classifichiamo i virus in base al tipo di target che il virus cerca di infettare e il metodo di cui fa uso il virus per nascondersi dal rilevamento da parte degli utenti e dagli antivirus.

\subsubsection{In Base al Target da Attaccare}

\begin{description}
    \item[Boot sector infector] Infetta un master boot record o un boot record e si diffonde quando un sistema viene avviato dal disco contenente il virus.
    \item[File infector] Infetta i file che il sistema operativo o la shell considerano come eseguibili.
    \item[Macro virus] Infetta i file con macro o codice di scripting che viene interpretato da un'applicazione.
    \item[Virus multipartito] Infetta i file in diversi modi. In genere, il virus multipartito è in grado di infettare più tipi di file, quindi l'eliminazione totale del virus prevede l'intervento in tutti i potenziali luoghi di infezione.
\end{description}

\subsubsection{In Base alla Strategia di Camuffamento}

Una classificazione dei virus per strategia di camuffamento comprende le seguenti categorie:

\begin{description}
    \item[Virus criptato] Che utilizza la crittografia per occultare il proprio contenuto. Una porzione di virus crea una chiave crittografica casuale e cifra il resto del virus. La chiave viene memorizzata assieme al virus. Quando viene invocato un programma infetto, il virus utilizza la chiave casuale memorizzata per decifrare il virus. Quando il virus si replica, viene selezionata una chiave casuale diversa. Dal momento che buona parte del virus è cifrata con una chiave diversa per ogni istanza, non è possibile osservare un pattern di bit costante.
    
    \item[Virus furtivo] Dove l'intero virus, e non soltanto un payload, viene nascosto. Può utilizzare mutazione del codice, compressione o tecniche di rootkit per raggiungere tale scopo.
    
    \item[Virus polimorfo] Che nel corso della replicazione crea copie funzionalmente equivalenti, ma con pattern di bit nettamente diversi per eludere gli antivirus. In questo caso, la signature del virus sarà diversa per ogni copia. Per ottenere tale variazione, il virus può inserire in modo casuale istruzioni inutili oppure scambiare l'ordine di istruzioni indipendenti. Un approccio più efficace consiste nell'utilizzare la crittografia, come per i virus criptati. La porzione di virus responsabile della generazione delle chiavi e dell'esecuzione della cifratura/decifratura viene detta mutation engine. La stessa mutation engine viene modificata a ogni utilizzo.
    
    \item[Virus metamorfico] Che si ridefinisce completamente a ogni iterazione, utilizzando più tecniche di trasformazione, incrementando di conseguenza la complessità del rilevamento. I virus metamorfici possono cambiare il proprio comportamento così come l'aspetto.
\end{description}

\subsection{Macro Virus e di Scripting}

Un macro virus si attacca ai documenti e utilizza le funzionalità di programmazione macro dell'applicazione del documento per essere eseguito e propagarsi.

Risultano particolarmente minacciosi per una serie di ragioni:

\begin{enumerate}
    \item Un macro virus è indipendente dalla piattaforma, infatti essi infettano il contenuto attivo in applicazioni di uso comune come documenti della suite Office o documenti Adobe PDF. Qualsiasi piattaforma hardware e sistema operativo che supporti queste applicazioni può essere infettato.
    \item I macro virus infettano documenti, non porzioni eseguibili di codice e si diffondono con facilità in quanto i documenti generalmente vengono condivisi nel corso di un normale utilizzo.
    \item Dal momento che i macro virus infettano i documenti dell'utente, piuttosto che i programmi di sistema, i tradizionali controlli di accesso al file system hanno un'utilità limitata nel prevenire la loro diffusione, visto che si suppone che siano gli utenti a modificarli.
    \item I macro virus sono molto più facili da creare o modificare rispetto ai tradizionali virus eseguibili.
    \item I dettagli delle macro dipendono dall'applicazione che le interpreta, quindi colpiranno sempre i documenti per un'applicazione specifica.
\end{enumerate}

Un altro possibile obiettivo per malware di tipo macro virus è costituito dai documenti PDF di Adobe. Tali documenti possono supportare una serie di componenti integrati, inclusi Javascript e altri tipi di codice di scripting.

\section{I Worm}

Un worm è un programma che cerca attivamente altre macchine da infettare e poi ogni macchina infettata sferra, a sua volta, attacchi ad altre macchine.

\subsection{Meccanismi di Diffusione}

I programmi worm sfruttano le vulnerabilità del software nei programmi client o server per ottenere l'accesso a ogni nuovo sistema. Per replicarsi, un worm si serve di mezzi per accedere ai sistemi remoti. Fra questi vi sono i seguenti, la maggior parte dei quali è ancora oggi in uso:

\begin{description}
    \item[Posta elettronica o messaggistica istantanea] Un worm invia una copia di se stesso ad altri sistemi via e-mail o messaggistica istantanea, cosicché il suo codice venga eseguito quando l'e-mail o l'allegato viene ricevuto o visualizzato.
    
    \item[Condivisione di file] Un worm crea una copia di se stesso oppure infetta altri file idonei su un supporto rimovibile come una chiavetta USB; poi entra in esecuzione quando l'unità viene collegata a un altro sistema utilizzando il meccanismo di esecuzione automatica, sfruttando una qualche vulnerabilità del software, o quando un utente apre il file infetto sul sistema target.
    
    \item[Funzionalità di esecuzione remota] Un worm esegue una copia di se stesso su un altro sistema utilizzando un'esplicita funzione di esecuzione remota o sfruttando una falla di programma in un servizio di rete per sovvertirne le operazioni.
    
    \item[Funzionalità di trasferimento o accesso remoto ai file] Un worm utilizza un servizio di accesso remoto ai file o di trasferimento a un altro sistema per copiarsi da un sistema all'altro, dove gli utenti di quel sistema potrebbero poi eseguirlo.
    
    \item[Funzionalità di login remoto] Un worm accede a un sistema remoto come utente e poi utilizza dei comandi per copiare se stesso da un sistema all'altro, dove poi viene eseguito.
\end{description}

Di norma un worm presenta le stesse fasi di un virus informatico: dormiente, di propagazione, di attivazione e di esecuzione.

\subsection{Ricerca del Target}

La prima operazione nella fase di propagazione di un worm di rete consiste nel cercare altri sistemi da infettare, un processo noto come scanning o fingerprinting. Tipicamente, il codice del worm appena installato sulle macchine infette ripete lo stesso processo di scanning, fino a quando non viene creata un'ampia rete distribuita di macchine infette.

Un worm utilizza generalmente le seguenti strategie di scanning degli indirizzi di rete:

\begin{description}
    \item[Casuale] Ogni host compromesso analizza indirizzi casuali nello spazio degli indirizzi IP servendosi di un seme diverso.
    
    \item[Hit-list] L'attaccante redige anzitutto una lunga lista di macchine potenzialmente vulnerabili. Questo può essere un processo lento e prolungato per evitare che si scopra che è in corso un attacco. Una volta che la lista è stata stilata, l'attaccante inizia a infettare le macchine presenti nella lista. A ogni macchina infettata viene fornita una porzione della lista da scansionare. Questa strategia si traduce in un processo di scanning piuttosto rapido che potrebbe rendere difficile rilevare che l'infezione è in corso.
    
    \item[Topologica] Questo metodo utilizza le informazioni contenute nelle macchine vittime infettate per trovare altri host da scansionare.
    
    \item[Sottorete locale] Se un host può essere infettato alle spalle di un firewall, tale host poi cerca target nella propria rete locale.
\end{description}

\subsection{Propagazione del Worm}

La propagazione si sviluppa seguendo tre fasi. Nella fase iniziale il numero di host aumenta esponenzialmente. Dopo un po' gli host infetti sprecano tempo attaccando host già infetti, il che riduce il tasso d'infezione. Durante questa fase intermedia la crescita è approssimativamente lineare, ma il tasso di infezione è elevato. Quando la maggior parte dei computer vulnerabili è stata infettata, l'attacco entra in una fase finale lenta in cui il worm va alla ricerca degli host rimanenti di difficile identificazione.

\section{Vulnerabilità Lato Client}

Le vulnerabilità nel software, in particolare nelle applicazioni client come browser e plugin, rappresentano un obiettivo comune per gli attaccanti. L'approccio più comune per sfruttare queste vulnerabilità è il drive-by-download, una tecnica che consente il download e l'installazione di malware sul sistema dell'utente senza il suo consenso o la sua consapevolezza.

\subsection{Drive-by-Download}

Un attacco drive-by-download si verifica quando un utente visita una pagina web compromessa, che contiene codice malevolo. Questo codice sfrutta vulnerabilità nel browser o nei plugin installati per scaricare e installare malware. A differenza dei worm, che si diffondono attivamente, il malware in un attacco drive-by-download attende semplicemente che l'utente visiti la pagina infetta.

\subsection{Watering Hole}

Questa variante degli attacchi drive-by-download è particolarmente mirata. Gli attaccanti identificano i siti web che le loro vittime designate sono propense a visitare e quindi cercano vulnerabilità in questi siti. L'obiettivo è compromettere tali siti in modo che, quando la vittima li visita, il codice malevolo venga eseguito solo sul suo sistema, riducendo la possibilità che il sito venga scoperto.

\subsection{Malvertising}

Il malvertising è un'altra tecnica utilizzata per diffondere malware senza compromettere direttamente i siti web. Gli attaccanti acquistano spazi pubblicitari che contengono malware e li pubblicano su siti legittimi. Gli annunci malevoli possono essere progettati per attivarsi solo in determinati orari o per determinati gruppi di utenti, rendendo più difficile la loro identificazione e rimozione.

\subsection{Clickjacking}

Il clickjacking è una vulnerabilità sfruttata dagli attaccanti per manipolare le interazioni degli utenti con un'interfaccia web. Questo tipo di attacco mira a ingannare gli utenti, indirizzando i loro clic verso elementi invisibili o mascherati, piuttosto che verso gli oggetti legittimi a cui si stanno rivolgendo.

Gli attaccanti utilizzano tecniche che impiegano più livelli di elementi grafici, come iframe o oggetti HTML, rendendoli invisibili o posizionandoli in modo ingannevole sopra o sotto elementi legittimi. Ciò consente all'attaccante di dirottare i clic dell'utente.

\section{Trojan}

Un trojan è un programma, in apparenza utile, contenente codice nascosto che, se invocato, svolge alcune operazioni indesiderate o dannose.

Possono essere utilizzati per compiere in maniera indiretta azioni che l'attaccante non potrebbe svolgere direttamente. I trojan possono rientrare in uno di questi tre modelli:

\begin{enumerate}
    \item Proseguire nell'esecuzione del programma originale e svolgere l'attività malevola in parallelo.
    \item Proseguire nell'esecuzione del programma originale, ma modificarne alcune funzioni per svolgere attività dannose (ad esempio, un programma di login che raccoglie le password) o per mascherarne altre (ad esempio, una versione trojan horse di un programma che elenca i processi che non mostra determinati processi dannosi).
    \item Svolgere una funzione dannosa che sostituisce completamente la funzione del programma originale.
\end{enumerate}

Alcuni trojan evitano la richiesta di intervento da parte dell'utente sfruttando alcune vulnerabilità software che consentono la loro installazione ed esecuzione in automatico. A differenza dei worm, i trojan non si replicano.

\section{Payload di un Malware}

Quando un malware viene attivato su un sistema target, l'attenzione si sposta sul suo payload, cioè le azioni specifiche che il malware eseguirà. Molti malware sono progettati per eseguire attività malevole a beneficio dell'attaccante, come la distruzione dei dati, il furto di informazioni o il danneggiamento di hardware e software. Alcuni payload possono restare latenti fino all'attivazione in base a specifiche condizioni, note come bombe logiche.

\subsection{Distruzione dei Dati}

La distruzione dei dati è uno dei payload più antichi e pericolosi. Un esempio è il virus Chernobyl (1998), che infettava i file eseguibili su Windows 95 e 98. Una volta attivato, cancellava il primo megabyte del disco rigido, danneggiando gravemente il file system. Un altro caso è il worm Klez (2001), che si diffondeva via e-mail e, in certe date, cancellava i file sui sistemi infetti.

\subsection{Ransomware}

Una variante ancora più dannosa è il ransomware, un malware che cripta i dati dell'utente e richiede un pagamento (di solito in criptovaluta) per fornire la chiave di decrittazione. Versioni più recenti di ransomware utilizzano chiavi di crittografia così avanzate che non possono essere facilmente violate, costringendo le vittime a pagare per recuperare i propri file.

\subsubsection{WannaCry}

Un esempio significativo di ransomware è WannaCry (2017), che ha infettato migliaia di sistemi in tutto il mondo, criptando i file e richiedendo un riscatto in Bitcoin per il loro recupero. Questo attacco ha colpito gravemente organizzazioni in diversi settori, causando perdite economiche significative e attirando molta attenzione mediatica. WannaCry ha sfruttato una vulnerabilità nei sistemi Windows e ha sottolineato l'importanza di backup regolari e di una strategia di disaster recovery.

\subsection{Bombe Logiche}

Le bombe logiche sono un altro tipo di payload comune nei malware. Sono porzioni di codice programmato per attivarsi quando si verificano determinate condizioni, come una specifica data, l'uso di un determinato software o l'accesso da parte di un particolare utente. Quando queste condizioni vengono soddisfatte, la bomba logica ``esplode'', causando la cancellazione di dati, l'arresto del sistema o altri danni.

\section{Agenti di Attacco}

\subsection{Bot}

Un bot (o zombie) assume segretamente il controllo di un altro computer collegato a Internet per poi utilizzarlo per lanciare attacchi che sono difficili da ricondurre al creatore del malware stesso.

\subsection{Botnet}

Il bot viene in genere installato su centinaia o migliaia di computer appartenenti a terzi ignari. I sistemi compromessi comprendono PC, server e dispositivi embedded quali router o videocamere di sorveglianza. L'insieme dei bot è spesso in grado di agire in modo coordinato; una rete di questo tipo prende il nome di botnet. Questo tipo di payload compromette l'integrità e la disponibilità del sistema infettato.

\subsection{Uso dei Bot}

I bot vengono generalmente utilizzati per i seguenti attacchi:

\begin{description}
    \item[Attacchi DDoS] Le botnet vengono impiegate per sovraccaricare un server o una rete con un flusso massiccio di traffico, causando l'interruzione del servizio per gli utenti legittimi.
    
    \item[Spamming] Grazie a una botnet, gli attaccanti possono inviare enormi quantità di e-mail di spam, utilizzando i bot per aggirare i controlli anti-spam e moltiplicare la portata dell'attacco.
    
    \item[Sniffing del traffico] I bot possono essere dotati di sniffer per intercettare dati sensibili (come credenziali) trasmessi in chiaro su una rete. Questo consente agli attaccanti di rubare informazioni critiche.
    
    \item[Keylogging] Se i dati sono trasmessi in modo criptato (ad esempio, via HTTPS), lo sniffing potrebbe non essere efficace. Tuttavia, i bot possono installare keylogger per catturare tutto ciò che viene digitato sulla tastiera, come nomi utente e password.
    
    \item[Diffusione di malware] Le botnet possono essere utilizzate per diffondere altri tipi di malware, come worm o virus, sfruttando la capacità dei bot di scaricare ed eseguire file dannosi su larga scala.
    
    \item[Installazione di adware] Per ottenere guadagni finanziari, le botnet possono generare clic automatizzati su pubblicità online. I bot possono anche reindirizzare le pagine iniziali dei browser delle macchine compromesse per incrementare ulteriormente i clic sugli annunci.
    
    \item[Attacchi a reti IRC] Le botnet possono attaccare le reti di chat IRC attraverso attacchi ``clone'', dove migliaia di bot si connettono simultaneamente alla rete bersaglio, causando disservizi simili a un DDoS.
    
    \item[Manipolazione di sondaggi e giochi online] Poiché ogni bot ha un indirizzo IP unico, le botnet possono essere usate per manipolare i risultati di sondaggi online o giochi, con migliaia di voti falsi che sembrano provenire da utenti reali.
\end{description}

\subsection{Controllo Remoto}

La funzione di controllo remoto è l'aspetto chiave che differenzia un bot da un worm. Un worm è auto-propagante e agisce autonomamente, mentre un bot viene gestito tramite un'infrastruttura di command-and-control (CnC), che consente agli attaccanti di controllare il bot in modo diretto o periodico, in base alla connessione del bot alla rete.

\subsubsection{Evoluzione del Controllo Remoto}

\begin{description}
    \item[Inizio tramite IRC] I primi bot utilizzavano server IRC per comunicare. Tutti i bot si collegavano a un canale IRC e interpretavano i messaggi ricevuti come comandi.
    
    \item[Uso di HTTP e altri protocolli] Le botnet moderne evitano l'uso di IRC per minimizzare i rischi di individuazione e utilizzano protocolli come HTTP per comunicazioni più nascoste. Alcune utilizzano anche meccanismi peer-to-peer per distribuire il controllo, evitando punti di vulnerabilità singoli.
\end{description}

\subsubsection{Tecniche Avanzate di Protezione dei CnC}

\begin{description}
    \item[Generazione automatica di nomi di dominio] Malware più recenti usano algoritmi per generare automaticamente numerosi nomi di dominio. Questo consente agli attaccanti di cambiare rapidamente server di controllo in caso di compromissione, rendendo difficile per le autorità sequestrare tutti i domini.
    
    \item[Fast-flux DNS] Questa tecnica cambia rapidamente gli indirizzi IP associati a un dominio, utilizzando server proxy all'interno della botnet stessa. I server CnC possono quindi alternarsi frequentemente, rendendo più difficile la loro individuazione e disattivazione.
\end{description}

\subsubsection{Modalità Operative del Modulo di Controllo}

\begin{itemize}
    \item Il modulo di controllo può inviare comandi ai bot per eseguire procedure preconfigurate, come attacchi DDoS o furto di dati.
    \item Per maggiore flessibilità, il modulo può ordinare ai bot di scaricare ed eseguire nuovi file, trasformandoli in strumenti più generici per una varietà di attacchi.
    \item Inoltre, il modulo raccoglie i dati sottratti dai bot, che vengono poi sfruttati dall'attaccante.
\end{itemize}

\subsection{Contromisure}

Un'efficace contromisura contro una botnet è il sequestro o la disattivazione della sua infrastruttura CnC.

\section{Furto di Credenziali}

Il furto di credenziali tramite keylogger e spyware è una tecnica comune utilizzata dagli attaccanti per aggirare le protezioni crittografiche, come HTTPS o POP3S, che proteggono i dati durante la trasmissione.

\subsection{Keylogger}

Un keylogger è un malware che registra le sequenze di tasti digitate sulla tastiera di un computer infetto. Anche se le credenziali vengono inviate su canali cifrati, il keylogger cattura i dati mentre vengono inseriti, prima che vengano criptati. Poiché registra ogni sequenza di tasti, i keylogger moderni utilizzano meccanismi di filtraggio, restituendo solo le informazioni pertinenti, come login, password, o URL sensibili.

\subsubsection{Contromisure}

Per contrastare i keylogger, alcuni siti bancari e servizi online hanno adottato applet grafiche per l'inserimento delle password. Questi strumenti non si basano sul testo digitato con la tastiera, rendendo inefficaci i keylogger tradizionali.

\subsection{Spyware}

In risposta all'uso di contromisure grafiche, gli attaccanti hanno sviluppato spyware più avanzati, capaci di:

\begin{itemize}
    \item Monitorare la cronologia e l'attività del browser.
    \item Reindirizzare le richieste di siti web a pagine false controllate dall'attaccante.
    \item Modificare dinamicamente i dati scambiati tra il browser e siti web legittimi, rubando o alterando informazioni personali e finanziarie.
\end{itemize}

\subsubsection{Trojan Bancario Zeus}

Un esempio noto di spyware sofisticato è il Trojan Zeus, distribuito tramite il toolkit crimeware. Zeus combina funzionalità di keylogger e di manipolazione dei form web, rubando credenziali bancarie e modificando i dati inseriti dagli utenti su siti finanziari. Zeus viene tipicamente diffuso tramite email di spam o con attacchi drive-by-download su siti compromessi, infettando i dispositivi senza che l'utente se ne accorga.

\subsection{Phishing}

Il phishing è una tecnica di attacco che sfrutta l'ingegneria sociale per rubare credenziali e informazioni sensibili attraverso e-mail ingannevoli. Queste e-mail solitamente contengono un link a un sito web fraudolento, progettato per imitare pagine di login spingendo la vittima a inserire le proprie credenziali.

Gli attaccanti spesso usano un pretesto di urgenza, come l'imminente blocco dell'account, per convincere la vittima ad agire rapidamente. Una volta che l'utente fornisce i dettagli richiesti, gli attaccanti possono accedere all'account per sfruttarlo.

\subsubsection{Meccanismo del Phishing}

\begin{enumerate}
    \item \textbf{Email ingannevole:} L'attaccante invia una e-mail che sembra provenire da una fonte affidabile, contenente un link a un sito fraudolento o un modulo da compilare e restituire.
    \item \textbf{Sito web ingannevole:} Il sito web è progettato per sembrare autentico, ma è controllato dall'attaccante.
    \item \textbf{Furto d'identità:} Con abbastanza informazioni raccolte, l'attaccante può assumere l'identità della vittima, accedendo a conti bancari, ottenendo credito o compiendo altre azioni fraudolente.
\end{enumerate}

\subsubsection{Spam e Botnet}

Le e-mail di phishing vengono solitamente inviate a un vasto numero di destinatari tramite botnet. Anche se molte e-mail non colpiscono i destinatari corretti, una piccola percentuale di persone può essere ingannata, rendendo comunque l'attacco redditizio.

\subsubsection{Spear-phishing}

Una variante più mirata è il spear-phishing, in cui gli attaccanti selezionano con attenzione le vittime e creano e-mail personalizzate. Questo tipo di attacco è spesso utilizzato per spionaggio industriale o frodi finanziarie, come false autorizzazioni di trasferimenti bancari.

\section{Stealthing}

Con il termine stealthing si intendono le tecniche utilizzate dai malware per nascondere la propria presenza in un sistema infetto, rendendo difficile per il software antivirus rilevarlo e rimuoverlo.

\subsection{Backdoor}

Una backdoor (o trapdoor) è un accesso segreto a un sistema che consente a un utente autorizzato di bypassare le normali procedure di sicurezza.

\subsubsection{Tipologie e Utilizzi}

\begin{description}
    \item[Backdoor legittime] (o maintenance hooks), progettate dagli sviluppatori per semplificare il debug di software complessi. Possono consentire di ottenere privilegi speciali o saltare processi di autenticazione durante lo sviluppo.
    \item[Backdoor malevole] sfruttabili per ottenere accesso non autorizzato ai sistemi.
\end{description}

\subsubsection{Sicurezza e Prevenzione}

Implementare misure di sicurezza contro le backdoor richiede attenzione durante le fasi di sviluppo e aggiornamento del software. Alcuni approcci includono:

\begin{itemize}
    \item \textbf{Revisione del codice:} Analizzare il codice sorgente per identificare eventuali backdoor non autorizzate.
    \item \textbf{Test di penetrazione:} Condurre test di sicurezza per simulare attacchi e identificare vulnerabilità.
    \item \textbf{Monitoraggio delle connessioni di rete:} Tenere traccia delle connessioni in ingresso e uscita per rilevare attività sospette.
    \item \textbf{Utilizzo di strumenti di rilevamento delle anomalie:} Implementare soluzioni di sicurezza che possono identificare comportamenti anomali o non autorizzati nel sistema.
\end{itemize}

\subsection{Rootkit}

Un rootkit è un insieme di strumenti progettati per ottenere e mantenere un accesso segreto a un sistema informatico, spesso con privilegi di amministratore (o ``root''). Questi programmi nascondono la loro presenza e possono modificare il funzionamento del sistema operativo in modo da sfuggire alla rilevazione.

\subsubsection{Funzionalità e Obiettivi}

Il principale obiettivo di un rootkit è fornire all'attaccante il controllo completo sul sistema, consentendo azioni come:

\begin{itemize}
    \item Modifica o aggiunta di programmi e file: Un rootkit può installare ulteriori malware o alterare file esistenti.
    \item Monitoraggio dei processi: Può tenere traccia delle attività del sistema, compreso il monitoraggio delle comunicazioni di rete.
    \item Accesso da backdoor: Permette di accedere al sistema anche dopo un riavvio o una disconnessione.
\end{itemize}

\subsubsection{Come Nasconde la Propria Presenza}

I rootkit utilizzano diverse tecniche per nascondersi, rendendo difficile la loro identificazione:

\begin{enumerate}
    \item \textbf{Sovversione dei meccanismi di monitoraggio:} Modificano i sistemi di registrazione, monitoraggio e reporting di file e processi per non mostrare la loro presenza.
    \item \textbf{Alterazione dei risultati delle chiamate API:} Possono intercettare le chiamate delle API per mascherare le informazioni su processi e file associati.
\end{enumerate}

\subsubsection{Classificazione dei Rootkit}

I rootkit possono essere classificati in base a diverse caratteristiche:

\begin{description}
    \item[Persistente] Si attiva ad ogni avvio del sistema, memorizzando il codice in un archivio persistente (come il registro di sistema). Questi sono più facili da rilevare, ma possono essere progettati per rimanere nascosti.
    
    \item[Residente in memoria] Non ha codice persistente e quindi non sopravvive a un riavvio. È più difficile da rilevare perché risiede solo nella memoria.
    
    \item[Modalità utente] Intercetta le chiamate alle API e altera i risultati, come escludere i file di rootkit dalle liste dei file.
    
    \item[Modalità kernel] Intercetta le chiamate alle API di sistema a livello di kernel, nascondendo processi e file a livello più profondo.
    
    \item[Basato su macchina virtuale] Installa un monitor di macchina virtuale che esegue il sistema operativo in un ambiente virtuale. Il rootkit può quindi modificare gli stati del sistema senza essere rilevato.
    
    \item[Modalità esterna] Risiede al di fuori del sistema operativo, in modalità BIOS o di gestione del sistema, avendo accesso diretto all'hardware.
\end{description}